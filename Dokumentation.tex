\documentclass{article}
\usepackage[utf8]{inputenc}
\usepackage{hyperref}

\title{Dokumentation}
\author{Steffen Tuermer }
\date{September 2020}

\usepackage{natbib}
\usepackage{graphicx}

\begin{document}

\maketitle

\section{Themensammlung}
\subsection{Aufgaben:}

\subsubsection{Auswahl GPS-Modul}
Kriterien:
\begin{itemize}
\item	
Größe
\item	
Auffrisch-Rate
\item	
Baudrate
\item	
Empfindlichkeit
\item	
Leistungsaufnahme
\item	
Anzahl Channels (L1, L2, L5)
\item	
Art der Antenne und Positioitemnierung im Prototyp
\item	
Positionsgenauigkeit
\item	
Geschwindigkeitsgenauigkeit
\item	Schnittstelle Mikrocontroller
\item	Code Mikrocontroller
\item	Postprocessing der GPS Daten (Ephemeriedendaten?)
\end{itemize}

\subsection{Forschungsfragen:}

\begin{itemize}


\item	Interaktion mit Schnee:
\item	Störung durch Bedeckung mit Schnee
\item	Gewinnung von Schneeeigenschaften durch Analyse des GPS Signals
\item	Mutlipath Effekte durch Schnee
\item	Interaktion mit Sensoren:
\item	IMU (Gyro, Mag, Acc)
\item	Störung durch Rettungsreflektor (Recco, Pieps)
\item	Störun durch Gehäuse
\item	Outlook:
\item	Sinnhaftigkeit einer GPS – RTK Messung?
\item	Welche Probleme tretten auf, wie kann man diese lösen?

\end{itemize}

\subsection{Literatur:}
\begin{itemize}
\item	
GPS signal reception under snow cover: a pilot study establishing the potential usefulness of GPS in avalanche search andrescue operations \url{https://pubmed.ncbi.nlm.nih.gov/11990140/}
\item	
\url{https://www.researchgate.net/publication/253280455_GPS_Tracking_Performance_under_Avalanche_Deposited_Snow} GPS Tracking Performance under Avalanche Deposited Snow
\item	
Snow Water Equivalent of Dry Snow Derived From GNSS Carrier Phases \url{https://ieeexplore.ieee.org/document/8307768}
\item	
Retrieval of Snow Water Equivalent, Liquid Water Content, and Snow Height of Dry and Wet Snow by Combining GPS Signal Attenuation and Time Delay \url{https://agupubs.onlinelibrary.wiley.com/doi/full/10.1029/2018WR024431}
\end{itemize}
Vorschläge für Chip:
\begin{itemize}
\item	
\url{https://at.rs-online.com/web/p/gps-chip-und-gps-module/1793714?cm_mmc=AT-PLA-DS3A-_-google-_-CSS_AT_DE_Computertechnik_und_Peripherieger%C3%A4te_Whoop_ME-_-(AT:Whoop!)+GPS+Chip+und+GPS+Module-_-1793714&matchtype=&aud-827186183686:pla-333680702070&gclid=CjwKCAjwydP5BRBREiwA-qrCGnt3E3HH_9qw8EMhWAKEcSGWPeI81MjX6XkguFrurs_4tr8YxfDS0RoCawAQAvD_BwE&gclsrc=aw.ds}
\item	
\url{https://www.seeedstudio.com/blog/2019/11/06/arduino-gps-modules-which-one-to-use-guide-and-comparisons/}
\end{itemize}
\section{Zeitplan}

\section{GPS Implementation}

\subsection{Komponenten}

\begin{itemize}
\item Arduino Nano
\item ADAFRUIT Ultimate GPS Shield
\item SD-Module
\begin{itemize}
\item 32GB Micro SD
\end{itemize}
\end{itemize}

\subsection{GPS Grundlagen}
\subsubsection{NMEA}

Kommunikationsstandard zwischen Navigationssgeräten. Am häufigsten verwendet werden $\#$GPGGA und $\#$GPRMC. \url{http://aprs.gids.nl/nmea/}
Format GPRMC:\\
$GPRMC,hhmmss.ss,A,llll.ll,a,yyyyy.yy,a,x.x,x.x,ddmmyy,x.x,a*hh$\\
1    = UTC of position fix\\
2    = Data status (V=navigation receiver warning)\\
3    = Latitude of fix\\
4    = N or S\\
5    = Longitude of fix\\
6    = E or W\\
7    = Speed over ground in knots\\
8    = Track made good in degrees True\\
9    = UT date\\
10   = Magnetic variation degrees (Easterly var. subtracts from true course)\\
11   = E or W\\
12   = Checksum



$GPGGA,hhmmss.ss,llll.ll,a,yyyyy.yy,a,x,xx,x.x,x.x,M,x.x,M,x.x,xxxx*hh$
1    = UTC of Position
2    = Latitude
3    = N or S
4    = Longitude
5    = E or W
6    = GPS quality indicator (0=invalid; 1=GPS fix; 2=Diff. GPS fix)
7    = Number of satellites in use [not those in view]
8    = Horizontal dilution of position
9    = Antenna altitude above/below mean sea level (geoid)
10   = Meters  (Antenna height unit)
11   = Geoidal separation (Diff. between WGS-84 earth ellipsoid and
       mean sea level.  -=geoid is below WGS-84 ellipsoid)
12   = Meters  (Units of geoidal separation)
13   = Age in seconds since last update from diff. reference station
14   = Diff. reference station ID
15   = Checksum


\subsection{GPS-Modul Vergleich}

\section{Meetings}
\subsection{200908 Strombedarf Nano+UltimateGPS+SD}
Zur Messung des Strombedarf wurde das System mit einem Labornetzteil versorgt. Für den Arduino wird im Datenblatt eine Versorgungsspannung von 7-12$V$ empfohlen. Es wird der Strombedarf bei 4.5$V$ bis 10$V$ gemessen.\\
\subsubsection{Erwartete Werte}
Folgende Werte sind im Datenblatt der verwendeten Komponenten angeführt:
\begin{itemize}
\item
Ardino Nano: 25$mA$ im Normalbetrieb
\item
SD-Breakout: max 100mA pro Schreibvorgang
\item
Ultimate GPS: 20mA
\end{itemize}

\subsubsection{Gemessene Werte}
Folgende Variationen werden gemessen:
\begin{itemize}
\item{nur GPS}
\item{GPS+Log}
\item{GPS Frequency=1$Hz$/10$Hz$}
\item{fix/nofix}
\end{itemize}

\href{D:/Dropbox/Masterarbeit AvaRange GPS/Dokumentation/Strommessung.txt}{Messwerte}
\subsubsection{Ergebnis}

Es hat sich gezeigt, dass sowohl GPS Modul, als auch SD-Modul bei einer Versorgung bis 4.6V noch funktionieren. Allerdings ist der Stromverbrauch unter 5.5$V$ nicht konstant. Um sicherzustellen, dass der Arduino richtig arbeitete sollte er mit mindestens 5.5$V$ betrieben werden.\\
Die gemessenen Werte bei einer GPS-Frequenz von 1$Hz$ entsprechen etwa den erwarteten Werten. Wird die Frequenz auf 10$Hz$ erhöht steigt der Stromverbrauch von durchschnittlich 45$mA$ auf 55$mA$. Dies Werte wurden ohne eine Satellitenverbindung ermittelt. Ist eine Verbindung aufgebaut steigt der Verbrauch um weitere 3$mA$.\\
Der Stromverbrauch des SD-Modul ist bei diesen relativ niedrigen Frequenzen nicht erheblich.


\subsection{200825 Meeting Gerstmayr, Neurauther,Türmer}
Rene:\\
Zusammenfassung Ideen Masterarbeit:
\begin{itemize}
\item
Integration eines GPS Moduls:
\begin{itemize}
\item
Auswahl GPS Modul
\item
Inbetriebnahme
\item
Synchronisation mit IMUs
\end{itemize}

\item
Durchführung Experimente:

\begin{itemize}
\item
wie lange dauert die Positionsbestimmung
\item
Auswirkung Wartezeit
\item
Auswirkung Einhausung
\end{itemize}

\item
Fusionierung der Messdaten
\begin{itemize}
\item
Vergleich der Endpositon IMU vs GPS bei bekannter Startposition
\item
Korrektur der Endposition
\item
Literaturrecherche zu bstehenden Modellen
\item
Erstellen 2d Modell (Snowball in Ebene)
\end{itemize}
\end{itemize}


\subsection{200908 Meeting Gerstmayr, Neurauther, Neuhauser, Türmer, Fischer}

Michi:
\begin{itemize}
\item
Recco: Unterstützung zugesagt.
\item
Laserscan: 1. Scan durchgeführt. Blickwinkel geprüft.
\end{itemize}

JT: 

\begin{itemize}
\item
Nordkette: ofizielle Zusage, muss noch persönlich Abgesprochen werden.
\item
Ortung: am Besten Recco+miniLVS
\end{itemize}

Rene:
\begin{itemize}
\item
Prototyp gedruckt, IMU+SD, keine Zeitüberschreitung
\item
Laborversuche geplant(Translation+Drehungen), bis jetzt Freihand
\item
Testwürfel für Labor, kein Temperatursensor etc., Schnittstelle vorhanden,
\item
Erste Test Anfang Dezember
\item
PCB-Design
\end{itemize}

GPS:
\begin{itemize}
\item
Testabläufe
\item
Modulauswahl
\item
Stromverbrauch
\end{itemize}

\bibliographystyle{plain}
\bibliography{references}
\end{document}
